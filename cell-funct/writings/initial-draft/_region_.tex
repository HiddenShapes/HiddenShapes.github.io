\message{ !name(cell-cat.tex)}\documentclass[11pt,a4paper]{amsart}

\title{Morphisms of cellular automata}

\usepackage{dirtytalk}
\usepackage{color}
\usepackage{hyperref}
\usepackage{tikz-cd}
\usepackage{tikz}
\usepackage{etoolbox}

\makeatletter
\patchcmd{\@startsection}
  {\@afterindenttrue}
  {\@afterindentfalse}
  {}{}
  \makeatother

\usepackage{graphicx, nicefrac}
\let\rfb\reflectbox
\newcommand{\uglyfrac}[2]{\rfb{\color{red}\nicefrac{\rfb{#1}}{\rfb{#2}}}}

\theoremstyle{plain}
\newtheorem{THM}{Theorem}[section]
\newtheorem{PROP}[THM]{Proposition}
\newtheorem{LEMMA}[THM]{Lemma}
\newtheorem{CONJ}[THM]{Conjecture}
\newtheorem{COR}[THM]{Corollary}
\theoremstyle{definition}
\newtheorem{REM}[THM]{Remark}
\newtheorem*{REM*}{Remark}
\newtheorem{EX}[THM]{Example}
\newtheorem{DEF}[THM]{Definition}
\newtheorem*{DEF*}{Definition}

\DeclareMathOperator{\End}{End}
\DeclareMathOperator{\Mod}{Mod}
\DeclareMathOperator{\Hom}{Hom}
\DeclareMathOperator{\Aut}{Aut}
\DeclareMathOperator{\Obj}{Obj}
\DeclareMathOperator{\Id}{Id}
\DeclareMathOperator{\Sym}{Sym}
\DeclareMathOperator{\id}{id}
\DeclareMathOperator{\Cell}{\underline{\mathbb{C}ell}}

\newcommand{\de}{\delta}
\newcommand{\bC}{\mathbb{C}}
\newcommand{\bG}{\mathbb{G}}
\newcommand{\bN}{\mathbb{N}}
\newcommand{\bZ}{\mathbb{Z}}

\renewcommand\labelitemi{---}

\newcommand{\com}[1]{\textcolor{red}{#1}}


\begin{document}

\message{ !name(cell-cat.tex) !offset(-3) }


\maketitle

\section{Setup}

Let $S$ be a non-empty finite set of states and let $G$ be an non-empty graph. A \emph{Cellular Automaton} (CA) on $(G,S)$ is a function,
$$\de \colon S^{|V|} \rightarrow S^{|V|}.$$
We call an element of $S^{|V|}$ a \emph{configuration}, and call this function the \emph{global transition function}, it describes the dynamics of the automaton, mapping the current configuration (state of each vertex) to the next configuration. 

\begin{REM}
  \label{rem:contraints}
  Obviously, $\de$ should be constrained by $G$, in some way. But, for now, one of the good/bad things is that everything works for an arbitrary $\de \in \End(S^{|V|})$.
\end{REM}

For a given cellular automata $(G,S,\de)$, we consider a cateogry $\bC$ defined as follows. 
\begin{itemize}
\item $\Obj(\bC) = S^{|V|}$
\item $\Hom(a,b) = \{n \in \mathbb{N} \mid \de^n(a)=b \}.$
\end{itemize}
A nice feature of this construction is that composition is simply given by addition.

Our real object of interest is the 2-category $\Cell$  of such categories (i.e. categories that arise from cellular automata). In particular, we consider $\bG$, the category arsing from the game of life, and ask the following explorative question: can we construct a non-trivial functor $F \colon \bG \to \bC$? An initial challenge in answering this question is finding the right characterisation of \say{non-trivial}. This is the objective of the next section, however, before that there is some abstract nonsense left to do.

\begin{DEF}
  \label{def:equivalent-automata}
  Two cellular automata are considered \emph{equivalent} if their corresponding categories are. 
\end{DEF}


\begin{DEF}
  \label{def:dead-automata}
  A cellular automata is called \emph{dead} if it is equivalent to a cellular automata with only one state. 
\end{DEF}

\begin{PROP}
  \label{prop:dead-zero}
  Any dead cellular automata is a zero object in $\Cell$. \com{This could be incorrect, but if that's true -- $\bC$ should be changed, see the discussion in Remark~\ref{rem:change-cat}.} Therefore dead cellular automata are unique up to unique isomorphism and we may speak of \href{https://ncatlab.org/nlab/show/generalized+the}{the} dead cellular automata, which we denote $\bot$.
\end{PROP}


\section{Trivial examples}

Obviously there are two \say{formal} trivial examples: 
\begin{align*}
  \textbf{0} \colon \bG \to \bot \quad \text{and} \quad \Id \colon \bG \to \bG. 
\end{align*}
For a given configuration $a \in \Obj(\bG)$, let $A_1(a)$ denote the set of configurations $b$ such that $\de{b}=a$
For a given configuration $a \in \Obj(\bG)$, we consider the following sets,
\begin{align*}
A_k(a) &= \{b \in \Obj(\bG) \mid \de^k = a\} \\ A(a) &= \{b \in \Obj(\bG) \mid \exists n \in \mathbb{N}, \de^n = a\}
\end{align*}
Vassili's description of how to make a different kind of morphism doesn't quite work 

\begin{REM}
  \label{rem:change-cat}
  lipsum
\end{REM}


\end{document}
\message{ !name(cell-cat.tex) !offset(-71) }
